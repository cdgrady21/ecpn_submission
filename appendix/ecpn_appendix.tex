% Options for packages loaded elsewhere
\PassOptionsToPackage{unicode}{hyperref}
\PassOptionsToPackage{hyphens}{url}
%
\documentclass[
]{article}
\usepackage{amsmath,amssymb}
\usepackage{lmodern}
\usepackage{float}
\usepackage{iftex}
\ifPDFTeX
  \usepackage[T1]{fontenc}
  \usepackage[utf8]{inputenc}
  \usepackage{textcomp} % provide euro and other symbols
\else % if luatex or xetex
  \usepackage{unicode-math}
  \defaultfontfeatures{Scale=MatchLowercase}
  \defaultfontfeatures[\rmfamily]{Ligatures=TeX,Scale=1}
\fi
% Use upquote if available, for straight quotes in verbatim environments
\IfFileExists{upquote.sty}{\usepackage{upquote}}{}
\IfFileExists{microtype.sty}{% use microtype if available
  \usepackage[]{microtype}
  \UseMicrotypeSet[protrusion]{basicmath} % disable protrusion for tt fonts
}{}
\makeatletter
\@ifundefined{KOMAClassName}{% if non-KOMA class
  \IfFileExists{parskip.sty}{%
    \usepackage{parskip}
  }{% else
    \setlength{\parindent}{0pt}
    \setlength{\parskip}{6pt plus 2pt minus 1pt}}
}{% if KOMA class
  \KOMAoptions{parskip=half}}
\makeatother
\usepackage{xcolor}
\usepackage[margin=1in]{geometry}
\usepackage{color}
\usepackage{fancyvrb}
\newcommand{\VerbBar}{|}
\newcommand{\VERB}{\Verb[commandchars=\\\{\}]}
\DefineVerbatimEnvironment{Highlighting}{Verbatim}{commandchars=\\\{\}}
% Add ',fontsize=\small' for more characters per line
\usepackage{framed}
\definecolor{shadecolor}{RGB}{248,248,248}
\newenvironment{Shaded}{\begin{snugshade}}{\end{snugshade}}
\newcommand{\AlertTok}[1]{\textcolor[rgb]{0.94,0.16,0.16}{#1}}
\newcommand{\AnnotationTok}[1]{\textcolor[rgb]{0.56,0.35,0.01}{\textbf{\textit{#1}}}}
\newcommand{\AttributeTok}[1]{\textcolor[rgb]{0.77,0.63,0.00}{#1}}
\newcommand{\BaseNTok}[1]{\textcolor[rgb]{0.00,0.00,0.81}{#1}}
\newcommand{\BuiltInTok}[1]{#1}
\newcommand{\CharTok}[1]{\textcolor[rgb]{0.31,0.60,0.02}{#1}}
\newcommand{\CommentTok}[1]{\textcolor[rgb]{0.56,0.35,0.01}{\textit{#1}}}
\newcommand{\CommentVarTok}[1]{\textcolor[rgb]{0.56,0.35,0.01}{\textbf{\textit{#1}}}}
\newcommand{\ConstantTok}[1]{\textcolor[rgb]{0.00,0.00,0.00}{#1}}
\newcommand{\ControlFlowTok}[1]{\textcolor[rgb]{0.13,0.29,0.53}{\textbf{#1}}}
\newcommand{\DataTypeTok}[1]{\textcolor[rgb]{0.13,0.29,0.53}{#1}}
\newcommand{\DecValTok}[1]{\textcolor[rgb]{0.00,0.00,0.81}{#1}}
\newcommand{\DocumentationTok}[1]{\textcolor[rgb]{0.56,0.35,0.01}{\textbf{\textit{#1}}}}
\newcommand{\ErrorTok}[1]{\textcolor[rgb]{0.64,0.00,0.00}{\textbf{#1}}}
\newcommand{\ExtensionTok}[1]{#1}
\newcommand{\FloatTok}[1]{\textcolor[rgb]{0.00,0.00,0.81}{#1}}
\newcommand{\FunctionTok}[1]{\textcolor[rgb]{0.00,0.00,0.00}{#1}}
\newcommand{\ImportTok}[1]{#1}
\newcommand{\InformationTok}[1]{\textcolor[rgb]{0.56,0.35,0.01}{\textbf{\textit{#1}}}}
\newcommand{\KeywordTok}[1]{\textcolor[rgb]{0.13,0.29,0.53}{\textbf{#1}}}
\newcommand{\NormalTok}[1]{#1}
\newcommand{\OperatorTok}[1]{\textcolor[rgb]{0.81,0.36,0.00}{\textbf{#1}}}
\newcommand{\OtherTok}[1]{\textcolor[rgb]{0.56,0.35,0.01}{#1}}
\newcommand{\PreprocessorTok}[1]{\textcolor[rgb]{0.56,0.35,0.01}{\textit{#1}}}
\newcommand{\RegionMarkerTok}[1]{#1}
\newcommand{\SpecialCharTok}[1]{\textcolor[rgb]{0.00,0.00,0.00}{#1}}
\newcommand{\SpecialStringTok}[1]{\textcolor[rgb]{0.31,0.60,0.02}{#1}}
\newcommand{\StringTok}[1]{\textcolor[rgb]{0.31,0.60,0.02}{#1}}
\newcommand{\VariableTok}[1]{\textcolor[rgb]{0.00,0.00,0.00}{#1}}
\newcommand{\VerbatimStringTok}[1]{\textcolor[rgb]{0.31,0.60,0.02}{#1}}
\newcommand{\WarningTok}[1]{\textcolor[rgb]{0.56,0.35,0.01}{\textbf{\textit{#1}}}}
\usepackage{graphicx}
\makeatletter
\def\maxwidth{\ifdim\Gin@nat@width>\linewidth\linewidth\else\Gin@nat@width\fi}
\def\maxheight{\ifdim\Gin@nat@height>\textheight\textheight\else\Gin@nat@height\fi}
\makeatother
% Scale images if necessary, so that they will not overflow the page
% margins by default, and it is still possible to overwrite the defaults
% using explicit options in \includegraphics[width, height, ...]{}
\setkeys{Gin}{width=\maxwidth,height=\maxheight,keepaspectratio}
% Set default figure placement to htbp
\makeatletter
\def\fps@figure{htbp}
\makeatother
\setlength{\emergencystretch}{3em} % prevent overfull lines
\providecommand{\tightlist}{%
  \setlength{\itemsep}{0pt}\setlength{\parskip}{0pt}}
\setcounter{secnumdepth}{-\maxdimen} % remove section numbering
\ifLuaTeX
  \usepackage{selnolig}  % disable illegal ligatures
\fi
\IfFileExists{bookmark.sty}{\usepackage{bookmark}}{\usepackage{hyperref}}
\IfFileExists{xurl.sty}{\usepackage{xurl}}{} % add URL line breaks if available
\urlstyle{same} % disable monospaced font for URLs
\hypersetup{
  pdftitle={How Contact Can Promote Societal Change Amid Intergroup Conflict: An Intergroup Contact Field Experiment in Nigeria - Supplementary Information},
  hidelinks,
  pdfcreator={LaTeX via pandoc}}

\title{How Contact Can Promote Societal Change Amid Intergroup Conflict:
An Intergroup Contact Field Experiment in Nigeria - Supplementary
Information}
\author{}
\date{\vspace{-2.5em}}

\begin{document}
\maketitle

{
\setcounter{tocdepth}{2}
\tableofcontents
}
\hypertarget{supplementary-information}{%
\section{Supplementary Information}\label{supplementary-information}}

\hypertarget{appendix-a-randomization-inference-and-bootstrapping}{%
\subsection{Appendix A: Randomization Inference and
Bootstrapping}\label{appendix-a-randomization-inference-and-bootstrapping}}

Randomization inference and bootstrapping are nonparametric methods to
generate \(p\)-values (randomization inference) and confidence intervals
(bootstrapping). With \emph{randomization inference}, we first shuffle
the treatment variable to break the relationship between treatment and
outcomes. Next we regress outcomes on treatment using our regression
equation and store the resulting coefficient. Lastly, we repeat that
process 10,000 times to create the distribution of coefficients we would
observe if treatment had no effect on outcomes -- the null hypothesis.
Our \(p\)-value is the proportion of the null distribution that is
greater than or equal to our observed coefficient.

\emph{Bootstrapping} for standard errors is similar, but instead of
shuffling the treatment indicator we resample units with replacement. By
resampling with replacement, we create the empirical distribution of our
data and the range of possible treatment effects we might observe if we
repeated the experiment 10,000 times. The treatment effect at the 2.5th
percentile and at the 97.5th percentile are equivalent to a 95\%
confidence interval .

In each of these procedures, we mimic our randomization process by
randomizing/resampling the intervention to communities in site-level
clusters and within state blocks. This means that both communities in an
implementation site (farmers and pastoralists) will always be
treated/sampled together and that assignment to the intervention and
resampling are conducted separately in Nassarawa and Benue, just as the
intervention was assigned in this study. This procedure ensures that our
null distribution (for \(p\)-values) is created by randomizing the
intervention between exchangeable units and that our empirical
distribution (for confidence intervals) is created by resampling units
as they were sampled.

\begin{center}\rule{0.5\linewidth}{0.5pt}\end{center}

\hypertarget{appendix-b-robustness-checks-for-community-analysis}{%
\subsection{Appendix B: Robustness checks for community
analysis}\label{appendix-b-robustness-checks-for-community-analysis}}

\hypertarget{survey-outcomes}{%
\subsubsection{Survey outcomes}\label{survey-outcomes}}

These tables shows results with different ways of making indices
(additive vs inverse-covariance weighted), different models for
estimating effects (differencing vs controlling-for), and different ways
of coding count variables (raw vs ranked). Each table is an outcome.
Rows are results for different ways of creating the outcomes. Columns
show the coefficient from OLS regression, true p-value from
randomization inference, and a binary ``base'' indicator showing which
method was used in the paper.

The base method is always inverse-covariance weighted indices; the
estimation method is controlling-for unless the baseline difference
between the treatment and control groups is 0.20 standard deviations or
more; the base method of handling count variables is dense rank. Only
contact outcomes use count variables, only survey outcomes have a
baseline and an endline and are measured with indices.

\begin{table}[H]
\begin{center}
\label{tab:attitude_tab}
\caption{\textbf{Community Attitudes.} Effect of ECPN on attitudes using alternative methods of estimation and index construction. The first column shows coefficients from OLS regression, the second column shows $p$-values from randomization inference, and the third column shows which method was used in the paper.}
\smallskip

\begin{tabular}{l|r|r|r}
\hline
  & coefficient & p-value & base\\
\hline
Controlling-for \& ICW & 0.116 & 0.045 & 1\\
\hline
Controlling-for \& Additive & 0.093 & 0.038 & 0\\
\hline
Differencing \& ICW & 0.100 & 0.145 & 0\\
\hline
Differencing \& Additive & 0.073 & 0.116 & 0\\
\hline
\end{tabular}


\end{center}
\end{table}

\begin{table}[H]
\begin{center}
\label{tab:security_tab}
\caption{\textbf{Community Perceptions of Security} Effect of ECPN on perceptions of security using alternative methods of estimation and index construction. The first column shows coefficients from OLS regression, the second column shows $p$-values from randomization inference, and the third column shows which method was used in the paper.}
\smallskip

\begin{tabular}{l|r|r|r}
\hline
  & coefficient & p-value & base\\
\hline
Controlling-for \& ICW & 0.098 & 0.032 & 0\\
\hline
Controlling-for \& Additive & -0.010 & 0.593 & 0\\
\hline
Differencing \& ICW & 0.159 & 0.020 & 1\\
\hline
Differencing \& Additive & 0.054 & 0.213 & 0\\
\hline
\end{tabular}


\end{center}
\end{table}

\begin{table}[H]
\begin{center}
\label{tab:contact_tab}
\caption{\textbf{Community Contact} Effect of ECPN on contact using alternative methods of estimation, index construction, and measuring count variables. The first column shows coefficients from OLS regression, the second column shows $p$-values from randomization inference, and the third column shows which method was used in the paper.}
\smallskip

\begin{tabular}{l|r|r|r}
\hline
  & coefficient & p-value & base\\
\hline
Controlling-for \& ICW \& Ranks & 0.013 & 0.424 & 0\\
\hline
Controlling-for \& Additive \& Ranks & 0.003 & 0.422 & 0\\
\hline
Differencing \& ICW \& Ranks & 0.138 & 0.060 & 1\\
\hline
Differencing \& Additive \& Ranks & 0.015 & 0.182 & 0\\
\hline
Controlling-for \& ICW \& Categories & 0.017 & 0.377 & 0\\
\hline
Controlling-for \& ICW \& Raw & -0.020 & 0.602 & 0\\
\hline
Differencing \& ICW \& Categories & 0.120 & 0.064 & 0\\
\hline
Differencing \& ICW \& Raw & 0.071 & 0.207 & 0\\
\hline
\end{tabular}


\end{center}
\end{table}

\begin{table}[H]
\begin{center}
\label{tab:percExp_tab}
\caption{\textbf{Community Contact Willingness (Percent Experiment)} Effect of ECPN on willingness to have contact with the outgroup using alternative methods of estimation. The first column shows coefficients from OLS regression, the second column shows $p$-values from randomization inference, and the third column shows which method was used in the paper.}
\smallskip

\begin{tabular}{l|r|r|r}
\hline
  & coefficient & p-value & base\\
\hline
Controlling-for & 0.093 & 0.051 & 0\\
\hline
Differencing & 0.062 & 0.239 & 1\\
\hline
\end{tabular}


\end{center}
\end{table}

\begin{table}[H]
\begin{center}
\label{tab:endExp_tab}
\caption{\textbf{Community Endorsement Experiment} Effect of ECPN on endorsement experiment using alternative methods of estimation. The first column shows coefficients from OLS regression, the second column shows $p$-values from randomization inference, and the third column shows which method was used in the paper.}
\smallskip

\begin{tabular}{l|r|r|r}
\hline
  & coefficient & p-value & base\\
\hline
Controlling-for & 0.103 & 0.158 & 0\\
\hline
Differencing & 0.123 & 0.212 & 1\\
\hline
\end{tabular}


\end{center}
\end{table}

\begin{table}[H]
\begin{center}
\label{tab:pgg_tab}
\caption{\textbf{Community Public Goods Game} Effect of ECPN on probability of donating and on donation amount. The first column shows coefficients from OLS regression and the second column shows $p$-values from randomization inference.}
\smallskip

\begin{tabular}{l|r|r}
\hline
  & coefficient & p-value\\
\hline
Donation (binary) & 0.022 & 0.294\\
\hline
Donation amount & -35.124 & 0.852\\
\hline
\end{tabular}


\end{center}
\end{table}

\hypertarget{behvioral-outcomes}{%
\subsubsection{Behvioral outcomes}\label{behvioral-outcomes}}

Enumerators (for surveys and for behavioral observation) were not
informed of a community's treatment status, but two of the enumerators
for behavioral observations interacted with the research team frequently
and may have intuited the study's hypotheses. They also only made
observations at treatment sites. If their intuition of program aims
affected their observations (or if their observations became more
positive over time for other reasons), our estimates of the treatment
effect could be inflated.

We first checked the change over time in outcomes they reported.
Descriptively, the two senior enumerators observed \emph{less}
interaction in treatment sites over time, suggesting that they were not
influenced by knowledge of the study's hypotheses. As a robustness
check, we then repeated the main analysis but removed observations from
these two enumerators. Coefficients with their data removed are very
similar to coefficients with their data included; \(p\)-values are
higher mainly as a result of fewer observations (and therefore degrees
of freedom).

\begin{table}[H]
\begin{center}
\label{tab:behObs_enumsOnly_tab}
\caption{\textbf{Robustness check for behavioral observations} This table shows the baseline-endline change in outcomes for the two senior enumerators who may have intuited the study's hypotheses. The first column shows coefficients from OLS regression, the second column shows $p$-values. The row names show outcomes (pastoralists in the market, farmers in the market, and outgroup attending events). The two enumerators reported decreases in market outcomes in treatment sites and no change to event outcomes in treatment sites, suggesting that they were not influenced by knowledge of the study's hypotheses.}
\smallskip

\begin{tabular}{l|r|r}
\hline
  & coefficient & p-value\\
\hline
market-pastoralists\_enum & -4.4079254 & 0.1744306\\
\hline
market-farmers\_enum & -10.8863636 & 0.0161117\\
\hline
events-outgroup\_enum & 0.1363636 & 0.9929502\\
\hline
\end{tabular}


\end{center}
\end{table}

\begin{table}[H]
\begin{center}
\label{tab:behObs_enumsCheck_tab}
\caption{\textbf{Robustness check for behavioral observations} This table shows behavioral observation outcomes with two senior enumerators removed. The first column shows coefficients from OLS regression, the second column shows $p$-values. The row names show outcomes (pastoralists in the market, farmers in the market, and outgroup attending events) and subsets of the data (all data, data without two enumerators, and data from only those two enumerators).}
\smallskip

\begin{tabular}{l|r|r}
\hline
  & coefficient & p-value\\
\hline
market-pastoralists\_all & 12.532 & 0.001\\
\hline
market-pastoralists\_without & 10.285 & 0.103\\
\hline
market-farmers\_all & 3.339 & 0.242\\
\hline
market-farmers\_without & 1.286 & 0.456\\
\hline
events-outgroup\_all & 9.228 & 0.186\\
\hline
events-outgroup\_without & 6.333 & 0.136\\
\hline
\end{tabular}


\end{center}
\end{table}

\begin{center}\rule{0.5\linewidth}{0.5pt}\end{center}

\hypertarget{appendix-c-robustness-checks-for-individual-analysis}{%
\subsection{Appendix C: Robustness checks for individual
analysis}\label{appendix-c-robustness-checks-for-individual-analysis}}

These tables shows results with different ways of making indices
(additive vs inverse-covariance weighted), different models for
estimating effects (differencing vs controlling-for), and different ways
of coding count variables (raw vs ranked). Each table is an outcome.
Rows are results for different ways of creating the outcomes. Columns
show the coefficient from OLS regression, true p-value from
randomization inference, and a binary ``base'' indicator showing which
method was used in the paper.

The base method is always inverse-covariance weighted indices; the
estimation method is controlling-for unless the baseline difference
between the participants and control groups is 0.20 standard deviations
or more; the base method of handling count variables is dense rank. Only
contact outcomes use count variables, only survey outcomes have a
baseline and an endline and are measured with indices.

\begin{table}[H]
\begin{center}
\label{tab:attitude_tab_ind}
\caption{\textbf{Individual Attitudes.} Effect of ECPN on attitudes using alternative methods of estimation and index construction. The first column shows coefficients from OLS regression, the second column shows $p$-values from randomization inference, and the third column shows which method was used in the paper.}
\smallskip

\begin{tabular}{l|r|r|r}
\hline
  & coefficient & p-value & base\\
\hline
Non: Controlling-for \& ICW & 0.031 & 0.264 & 0\\
\hline
Part: Controlling-for \& ICW & 0.058 & 0.129 & 0\\
\hline
Non: Controlling-for \& Additive & 0.154 & 0.202 & 0\\
\hline
Part: Controlling-for \& Additive & 0.269 & 0.081 & 0\\
\hline
Non: Differencing \& ICW & 0.054 & 0.130 & 1\\
\hline
Part: Differencing \& ICW & 0.060 & 0.130 & 1\\
\hline
Non: Differencing \& Additive & 0.183 & 0.144 & 0\\
\hline
Part: Differencing \& Additive & 0.296 & 0.049 & 0\\
\hline
\end{tabular}


\end{center}
\end{table}

\begin{table}[H]
\begin{center}
\label{tab:security_tab_ind}
\caption{\textbf{Individual Perceptions of Security} Effect of ECPN on perceptions of security using alternative methods of estimation and index construction. The first column shows coefficients from OLS regression, the second column shows $p$-values from randomization inference, and the third column shows which method was used in the paper.}
\smallskip

\begin{tabular}{l|r|r|r}
\hline
  & coefficient & p-value & base\\
\hline
Non: Controlling-for \& ICW & -0.022 & 0.681 & 0\\
\hline
Part: Controlling-for \& ICW & -0.024 & 0.675 & 0\\
\hline
Non: Controlling-for \& Additive & 0.015 & 0.467 & 0\\
\hline
Part: Controlling-for \& Additive & -0.007 & 0.540 & 0\\
\hline
Non: Differencing \& ICW & 0.045 & 0.178 & 1\\
\hline
Part: Differencing \& ICW & 0.050 & 0.186 & 1\\
\hline
Non: Differencing \& Additive & 0.083 & 0.244 & 0\\
\hline
Part: Differencing \& Additive & 0.146 & 0.123 & 0\\
\hline
\end{tabular}


\end{center}
\end{table}

\begin{table}[H]
\begin{center}
\label{tab:contact_tab_ind}
\caption{\textbf{Individual Contact} Effect of ECPN on contact using alternative methods of estimation, index construction, and measuring count variables. The first column shows coefficients from OLS regression, the second column shows $p$-values from randomization inference, and the third column shows which method was used in the paper.}
\smallskip

\begin{tabular}{l|r|r|r}
\hline
  & coefficient & p-value & base\\
\hline
Non: Controlling-for \& ICW \& Ranks & -0.029 & 0.735 & 0\\
\hline
Part: Controlling-for \& ICW \& Ranks & 0.062 & 0.094 & 0\\
\hline
Non: Controlling-for \& Additive \& Ranks & -0.024 & 0.771 & 0\\
\hline
Part: Controlling-for \& Additive \& Ranks & 0.041 & 0.094 & 0\\
\hline
Non: Differencing \& ICW \& Ranks & 0.002 & 0.492 & 1\\
\hline
Part: Differencing \& ICW \& Ranks & 0.098 & 0.018 & 1\\
\hline
Non: Differencing \& Additive \& Ranks & -0.005 & 0.580 & 0\\
\hline
Part: Differencing \& Additive \& Ranks & 0.063 & 0.017 & 0\\
\hline
Non: Controlling-for \& ICW \& Categories & -0.045 & 0.764 & 0\\
\hline
Part: Controlling-for \& ICW \& Categories & 0.063 & 0.203 & 0\\
\hline
Non: Controlling-for \& ICW \& Raw & -0.023 & 0.737 & 0\\
\hline
Part: Controlling-for \& ICW \& Raw & 0.044 & 0.126 & 0\\
\hline
Non: Differencing \& ICW \& Categories & 0.017 & 0.407 & 0\\
\hline
Part: Differencing \& ICW \& Categories & 0.130 & 0.029 & 0\\
\hline
Non: Differencing \& ICW \& Raw & -0.002 & 0.531 & 0\\
\hline
Part: Differencing \& ICW \& Raw & 0.066 & 0.025 & 0\\
\hline
\end{tabular}


\end{center}
\end{table}

\begin{table}[H]
\begin{center}
\label{tab:pgg_tab_ind}
\caption{\textbf{Individual Publid Goods Game} Effect of ECPN on probability of donating and on donation amount. The first column shows coefficients from OLS regression and the second column shows $p$-values from randomization inference.}
\smallskip

\begin{tabular}{l|r|r}
\hline
  & coefficient & p-value\\
\hline
Non: Donation (binary) & 0.050 & 0.081\\
\hline
Part: Donation (binary) & 0.020 & 0.295\\
\hline
Non: Donation amount & -27.023 & 0.743\\
\hline
Part: Donation amount & -53.740 & 0.875\\
\hline
\end{tabular}


\end{center}
\end{table}

\begin{center}\rule{0.5\linewidth}{0.5pt}\end{center}

\hypertarget{appendix-d-balance-tests}{%
\subsection{Appendix D: Balance Tests}\label{appendix-d-balance-tests}}

\begin{table}[H]
\begin{center}
\label{tab:bal_obs_tab1}
\caption{\textbf{Balance: Observational Data All Outcomes}}
\smallskip

\begin{tabular}{l|r|r|r|r|r|r|r}
\hline
  & Control.strat & Treatment.strat & adj.diff.strat & adj.diff.null.sd.strat & std.diff.strat & z.strat & p.strat\\
\hline
Pastoralists\_index\_rank\_events & 35.415 & 25.141 & -10.275 & 12.377 & -0.585 & -0.830 & 0.406\\
\hline
Farmers\_index\_rank\_events & 32.978 & 36.182 & 3.204 & 7.104 & 0.303 & 0.451 & 0.652\\
\hline
Pastoralists\_index\_rank\_markets & 25.458 & 14.151 & -11.308 & 6.576 & -1.375 & -1.720 & 0.086\\
\hline
Farmers\_index\_rank\_markets & 24.417 & 25.250 & 0.834 & 5.699 & 0.101 & 0.146 & 0.884\\
\hline
\end{tabular}


\end{center}
\end{table}

\begin{table}[H]
\begin{center}
\label{tab:bal_obs_tab2}
\caption{\textbf{Balance: Observational Data Omnibus P-value}}
\smallskip

\begin{tabular}{l|r|r|r}
\hline
  & chisquare & df & p.value\\
\hline
strat & 6.494 & 4 & 0.165\\
\hline
\end{tabular}


\end{center}
\end{table}

\begin{table}[H]
\begin{center}
\label{tab:bal_svy_tab1}
\caption{\textbf{Balance: Survey Data All Outcomes}}
\smallskip

\begin{tabular}{l|r|r|r|r|r|r|r}
\hline
  & Control.strat & Treatment.strat & adj.diff.strat & adj.diff.null.sd.strat & std.diff.strat & z.strat & p.strat\\
\hline
Baseline\_Attitudes & 0.542 & 0.566 & 0.023 & 0.065 & 0.098 & 0.357 & 0.721\\
\hline
Baseline\_Perceptions\_of\_Security & -0.459 & -0.537 & -0.079 & 0.071 & -0.246 & -1.113 & 0.266\\
\hline
Baseline\_Contact & 0.496 & 0.336 & -0.159 & 0.104 & -0.585 & -1.528 & 0.127\\
\hline
Baseline\_Percent\_Experiment & 0.443 & 0.474 & 0.031 & 0.056 & 0.206 & 0.543 & 0.587\\
\hline
Baseline\_Endorsement\_Experiment & -0.212 & -0.250 & -0.038 & 0.169 & -0.067 & -0.225 & 0.822\\
\hline
\end{tabular}


\end{center}
\end{table}

\begin{table}[H]
\begin{center}
\label{tab:bal_svy_tab2}
\caption{\textbf{Balance: Survey Data Omnibus P-value}}
\smallskip

\begin{tabular}{l|r|r|r}
\hline
  & chisquare & df & p.value\\
\hline
strat & 6.302 & 5 & 0.278\\
\hline
\end{tabular}


\end{center}
\end{table}

\begin{center}\rule{0.5\linewidth}{0.5pt}\end{center}

\hypertarget{appendix-e-placebo-tests}{%
\subsection{Appendix E: Placebo tests}\label{appendix-e-placebo-tests}}

Several of our outcomes are survey self-reports, and self-reports could
be affected by factors other than the intervntion. For example, our
survey results are suspect if respondents in treatment communities
learned the ``correct'\,' answers better than respondents in control
communities (social desirability bias). If social desirability accounts
for the effect in survey self-reports, we would also expect differences
between treatment and control for other normatively desirable attitudes.

To test social desirability effects, we conduct a placebo analysis using
attitudes about violence as a placebo. Attitudes about violence are a
good candidate for this placebo because intergroup contact should not
affect general attitudes about violence, but respondents may feel social
pressure to answer violence questions in a desirable way. We measure
attitudes about violence with a six question index asking respondents if
it is always, sometimes, rarely, or never justified to use violence in
certain situations, such as retaliating against violence or bringing
criminals to justice.

Respondents in treatment communities might also express more positive
attitudes towards the outgroup if attitudes were becoming more tolerant
in treatment villages in a way that was unrelated to the intervention.
If attitudes towards any outgroup were becoming more tolerant in
treatment communities compared to control communities, we would expect
attitudes towards religious outgroups to improve more in treatment
communities than control communities. The contact intervention should
not affect attitudes towards people from other religions because the
farmers and pastoralists are often the same religion.

Respondents in treatment communities also might have had better access
to information, and that information changed their
attitudes/perceptions. To measure access to information, we use
frequency of radio listening. If the treatment communities increased
their amount of radio listening significantly more than control
communities, it is possible their attitudes/perceptions changed due to
information and not the contact intervention.

Coefficients come from OLS regression equation specified in the paper
(using state-level blocked fixed effects). P-values come from the
randomization inference described in the paper and in Appendix A; they
are one-sided ``greater-than'' p-values. The base method used in the
paper always constructs indices using inverse-covariance weighting; it
uses the controlling-for method of difference-in-differences estimation
when an outcome's baseline difference between treatment and control is
less than 0.20 standard deviations; it uses the differencing method when
the baseline difference is 0.20 standard deviations or larger.

\hypertarget{community.}{%
\subsubsection{Community.}\label{community.}}

\begin{table}[H]
\begin{center}
\label{tab:pl_vio_tab}
\caption{\textbf{Community Placebo: Attitudes towards violence index.} Effect of ECPN on placebo outcome using alternative methods of estimation and index construction. The first column shows coefficients from OLS regression, the second column shows $p$-values from randomization inference, and the third column shows which method was used in the paper.}
\smallskip

\begin{tabular}{l|r|r|r}
\hline
  & coefficient & p-value & base\\
\hline
Controlling-for \& ICW & 0.010 & 0.466 & 0\\
\hline
Controlling-for \& Additive & 0.004 & 0.441 & 0\\
\hline
Differencing \& ICW & -0.067 & 0.687 & 1\\
\hline
Differencing \& Additive & -0.027 & 0.679 & 0\\
\hline
\end{tabular}


\end{center}
\end{table}

\begin{table}[H]
\begin{center}
\label{tab:pl_vio_tab1}
\caption{\textbf{Community Placebo: Components of violence index.} Effect of ECPN on components of placebo index (attitudes towards violence) using alternative methods of estimation and index construction. The first column shows coefficients from OLS regression, the second column shows $p$-values from randomization inference, and the third column shows which method was used in the paper.}
\smallskip

\begin{tabular}{l|r|r|r}
\hline
  & coefficient & p-value & base\\
\hline
Bring criminals to justice: Controlling-for & 0.034 & 0.252 & 0\\
\hline
Bring criminals to justice: Differencing & -0.092 & 0.792 & 1\\
\hline
Defend ones group: Controlling-for & -0.026 & 0.659 & 1\\
\hline
Defend ones group: Differencing & -0.026 & 0.663 & 0\\
\hline
Defend ones religion: Controlling-for & -0.031 & 0.662 & 1\\
\hline
Defend ones religion: Differencing & -0.031 & 0.671 & 0\\
\hline
Force the government to change their policies: Controlling-for & 0.004 & 0.436 & 1\\
\hline
Force the government to change their policies: Differencing & -0.012 & 0.640 & 0\\
\hline
Maintain culture and traditions: Controlling-for & -0.004 & 0.558 & 1\\
\hline
Maintain culture and traditions: Differencing & -0.011 & 0.585 & 0\\
\hline
Retaliate against violence: Controlling-for & -0.007 & 0.658 & 1\\
\hline
Retaliate against violence: Differencing & 0.011 & 0.387 & 0\\
\hline
\end{tabular}


\end{center}
\end{table}

\begin{table}[H]
\begin{center}
\label{tab:pl_out_tab}
\caption{\textbf{Community Placebo: Trust towards religious outgroups.} Effect of ECPN on placebo outcome using alternative methods of estimation and index construction. The first column shows coefficients from OLS regression, the second column shows $p$-values from randomization inference, and the third column shows which method was used in the paper.}
\smallskip

\begin{tabular}{l|r|r|r}
\hline
  & coefficient & p-value & base\\
\hline
Controlling-for & 0.017 & 0.349 & 1\\
\hline
Differencing & -0.002 & 0.519 & 0\\
\hline
\end{tabular}


\end{center}
\end{table}

\begin{table}[H]
\begin{center}
\label{tab:pl_rad_tab}
\caption{\textbf{Community Placebo: Radio listening frequency.} Effect of ECPN on placebo outcome using alternative methods of estimation and index construction. The first column shows coefficients from OLS regression, the second column shows $p$-values from randomization inference, and the third column shows which method was used in the paper.}
\smallskip

\begin{tabular}{l|r|r|r}
\hline
  & coefficient & p-value & base\\
\hline
Controlling-for & 0.021 & 0.430 & 1\\
\hline
Differencing & 0.021 & 0.441 & 0\\
\hline
\end{tabular}


\end{center}
\end{table}

\hypertarget{individual}{%
\subsubsection{Individual}\label{individual}}

\begin{table}[H]
\begin{center}
\label{tab:pl_vio_ind}
\caption{\textbf{Individual Placebo: Attitudes towards violence index.} Effect of ECPN on placebo outcome using alternative methods of estimation and index construction. The first column shows coefficients from OLS regression, the second column shows $p$-values from randomization inference, and the third column shows which method was used in the paper.}
\smallskip

\begin{tabular}{l|r|r|r}
\hline
  & coefficient & p-value & base\\
\hline
Non: Controlling-for \& ICW & -0.057 & 0.801 & 0\\
\hline
Part: Controlling-for \& ICW & 0.013 & 0.413 & 0\\
\hline
Non: Controlling-for \& Additive & -0.163 & 0.770 & 0\\
\hline
Part: Controlling-for \& Additive & 0.022 & 0.441 & 0\\
\hline
Non: Differencing \& ICW & -0.033 & 0.642 & 1\\
\hline
Part: Differencing \& ICW & -0.016 & 0.549 & 1\\
\hline
Non: Differencing \& Additive & -0.058 & 0.580 & 0\\
\hline
Part: Differencing \& Additive & -0.023 & 0.508 & 0\\
\hline
\end{tabular}


\end{center}
\end{table}

\begin{table}[H]
\begin{center}
\label{tab:pl_vio_ind1}
\caption{\textbf{Individual Placebo: Components of violence index.} Effect of ECPN on components of placebo index (attitudes towards violence) using alternative methods of estimation and index construction. The first column shows coefficients from OLS regression, the second column shows $p$-values from randomization inference, and the third column shows which method was used in the paper.}
\smallskip

\begin{tabular}{l|r|r|r}
\hline
  & coefficient & p-value & base\\
\hline
Non: Bring criminals to justice: Controlling-for & -0.154 & 0.680 & 0\\
\hline
Part: Bring criminals to justice: Controlling-for & -0.031 & 0.535 & 0\\
\hline
Non: Bring criminals to justice: Differencing & -0.292 & 0.813 & 1\\
\hline
Part: Bring criminals to justice: Differencing & -0.494 & 0.933 & 1\\
\hline
Non: Defend ones group: Controlling-for & -0.078 & 0.616 & 1\\
\hline
Part: Defend ones group: Controlling-for & 0.039 & 0.440 & 1\\
\hline
Non: Defend ones group: Differencing & 0.034 & 0.447 & 0\\
\hline
Part: Defend ones group: Differencing & -0.021 & 0.525 & 0\\
\hline
Non: Defend ones religion: Controlling-for & -0.250 & 0.892 & 1\\
\hline
Part: Defend ones religion: Controlling-for & 0.141 & 0.276 & 1\\
\hline
Non: Defend ones religion: Differencing & -0.143 & 0.726 & 0\\
\hline
Part: Defend ones religion: Differencing & 0.067 & 0.378 & 0\\
\hline
Non: Force the government to change their policies: Controlling-for & -0.227 & 0.781 & 1\\
\hline
Part: Force the government to change their policies: Controlling-for & 0.031 & 0.457 & 1\\
\hline
Non: Force the government to change their policies: Differencing & -0.142 & 0.684 & 0\\
\hline
Part: Force the government to change their policies: Differencing & 0.013 & 0.496 & 0\\
\hline
Non: Maintain culture and traditions: Controlling-for & -0.190 & 0.797 & 1\\
\hline
Part: Maintain culture and traditions: Controlling-for & 0.024 & 0.443 & 1\\
\hline
Non: Maintain culture and traditions: Differencing & 0.022 & 0.453 & 0\\
\hline
Part: Maintain culture and traditions: Differencing & 0.161 & 0.261 & 0\\
\hline
Non: Retaliate against violence: Controlling-for & -0.091 & 0.639 & 1\\
\hline
Part: Retaliate against violence: Controlling-for & -0.080 & 0.601 & 1\\
\hline
Non: Retaliate against violence: Differencing & 0.176 & 0.254 & 0\\
\hline
Part: Retaliate against violence: Differencing & 0.156 & 0.286 & 0\\
\hline
\end{tabular}


\end{center}
\end{table}

\begin{table}[H]
\begin{center}
\label{tab:pl_out_ind}
\caption{\textbf{Individual Placebo: Trust towards religious outgroups.} Effect of ECPN on placebo outcome using alternative methods of estimation and index construction. The first column shows coefficients from OLS regression, the second column shows $p$-values from randomization inference, and the third column shows which method was used in the paper.}
\smallskip

\begin{tabular}{l|r|r|r}
\hline
  & coefficient & p-value & base\\
\hline
Non: Controlling-for & 0.178 & 0.208 & 1\\
\hline
Part: Controlling-for & -0.060 & 0.586 & 1\\
\hline
Non: Differencing & 0.140 & 0.283 & 0\\
\hline
Part: Differencing & -0.079 & 0.616 & 0\\
\hline
\end{tabular}


\end{center}
\end{table}

\begin{table}[H]
\begin{center}
\label{tab:pl_rad_ind}
\caption{\textbf{Individual Placebo: Radio listening frequency.} Effect of ECPN on placebo outcome using alternative methods of estimation and index construction. The first column shows coefficients from OLS regression, the second column shows $p$-values from randomization inference, and the third column shows which method was used in the paper.}
\smallskip

\begin{tabular}{l|r|r|r}
\hline
  & coefficient & p-value & base\\
\hline
Non: Controlling-for & 0.178 & 0.208 & 1\\
\hline
Part: Controlling-for & -0.060 & 0.586 & 1\\
\hline
Non: Differencing & 0.140 & 0.283 & 0\\
\hline
Part: Differencing & -0.079 & 0.616 & 0\\
\hline
\end{tabular}


\end{center}
\end{table}

\begin{center}\rule{0.5\linewidth}{0.5pt}\end{center}

\hypertarget{appendix-f-state-level-heterogeneous-effects}{%
\subsection{Appendix F: State-level heterogeneous
effects}\label{appendix-f-state-level-heterogeneous-effects}}

This looks at heterogeneous effects by state. This is a low-power
analysis. There are no significant differences in treatment effect by
state.

Coefficients and p-values estimating state-level heterogeneous effects
were calculated with robust OLS regression using site-level clusters.
The regression interacted the treatment indicator with the state
indicator. Benue was the reference category so this table shows
differences for Nasarawa.

\begin{table}[H]
\begin{center}
\label{tab:state_tab}
\caption{\textbf{State-level differences in community-level analysis.} There are not significant differences between the effect of the contact intervention by state. The first column shows coefficients from OLS regression, the second column shows $p$-values from OLS regression.}
\smallskip

\begin{tabular}{l|r|r}
\hline
  & coefficient & p-value\\
\hline
Attitudes & -0.176 & 0.315\\
\hline
Perceptions of security & 0.133 & 0.383\\
\hline
Contact & -0.104 & 0.469\\
\hline
Percent Experiment & -0.251 & 0.158\\
\hline
Endorsement Experiment & 0.191 & 0.488\\
\hline
PGG donation & -0.093 & 0.257\\
\hline
PGG amount & -84.807 & 0.442\\
\hline
\end{tabular}


\end{center}
\end{table}

\begin{center}\rule{0.5\linewidth}{0.5pt}\end{center}

\hypertarget{appendix-g-farmer-pastoralist-differences}{%
\subsection{Appendix G: Farmer-pastoralist
differences}\label{appendix-g-farmer-pastoralist-differences}}

We show demographic differences between farmers and pastoralists. We
also shows regressions using demographic characteristics as control
variables to confirm that accounting for these variables does not change
the study's results.

\begin{table}[H]
\begin{center}
\label{tab:demo_tab_overall1}
\caption{\textbf{Farmer and Pastoralist Demographics} This table shows demographic characteristics for farming groups and pastoralist groups}
\smallskip

\begin{tabular}{l|r|r|r|r|r|r|r|r|r|r}
\hline
farm\_pastor & age & female & income & fulani & christian & muslim & farming & pastoral & trading & radio\\
\hline
farmers & 36.31 & 0.43 & 42235.09 & 0 & 0.64 & 0.35 & 0.80 & 0.00 & 0.12 & 2.32\\
\hline
pastoralists & 34.66 & 0.46 & 30407.69 & 1 & 0.00 & 1.00 & 0.13 & 0.85 & 0.09 & 1.46\\
\hline
\end{tabular}


\end{center}
\end{table}

\begin{table}[H]
\begin{center}
\label{tab:demo_tab_state1}
\caption{\textbf{Farmer and Pastoralist Demographics by State} This table shows demographic characteristics for farming groups and pastoralist groups, separating respondents from Nasarawa and Benue}
\smallskip

\begin{tabular}{l|l|r|r|r|r|r|r|r|r|r|r}
\hline
farm\_pastor & state & age & female & income & fulani & christian & muslim & farming & pastoral & trading & radio\\
\hline
farmers & ben & 36.32 & 0.41 & 32521.10 & 0 & 0.98 & 0.00 & 0.97 & 0.00 & 0.02 & 2.50\\
\hline
farmers & nas & 36.30 & 0.44 & 48373.50 & 0 & 0.42 & 0.57 & 0.70 & 0.00 & 0.19 & 2.20\\
\hline
pastoralists & ben & 35.04 & 0.39 & 24947.04 & 1 & 0.00 & 1.00 & 0.04 & 0.88 & 0.10 & 1.55\\
\hline
pastoralists & nas & 34.40 & 0.51 & 34050.07 & 1 & 0.00 & 1.00 & 0.18 & 0.83 & 0.08 & 1.40\\
\hline
\end{tabular}


\end{center}
\end{table}

\begin{table}[H]
\begin{center}
\label{tab:comp_df1}
\caption{\textbf{Regressions with community-average demographic variables} This table shows coefficients from each survey regression either (1) the version in the paper that does not control for demographic characteristics, or (2) a regression that aggregated demographic characteristics to the community-level and included them as control variables. The coefficients are similar -- the sum of the coefficients when controlling for demographic characteristics is slightly larger.}
\smallskip

\begin{tabular}{l|r|r|r}
\hline
  & paper & check & difference\\
\hline
attitude & 0.1160135 & 0.0770404 & 0.0389731\\
\hline
in & 0.1590911 & 0.1681601 & -0.0090690\\
\hline
contactOnly & 0.1377790 & 0.1066938 & 0.0310852\\
\hline
rMean & 0.0621330 & -0.0080656 & 0.0701985\\
\hline
end\_exp & 0.1225162 & 0.2701144 & -0.1475982\\
\hline
pgp\_donate\_end & 0.0224200 & 0.0218400 & 0.0005800\\
\hline
pgp\_amount\_end & -35.1235007 & -32.4998039 & -2.6236968\\
\hline
total (excludes pgp\_amount) & 0.6199528 & 0.6357832 & -0.0158305\\
\hline
\end{tabular}


\end{center}
\end{table}

\begin{table}[H]
\begin{center}
\label{tab:comp_df_ind1}
\caption{\textbf{Regressions with individual demographic variables} This table shows coefficients from each survey regression either (1) the version in the paper that does not control for demographic characteristics, or (2) a regression that includes individual-level demographic characteristics as control variables. The coefficients are similar -- the sum of the coefficients when controlling for demographic characteristics is slightly smaller.}
\smallskip

\begin{tabular}{l|r|r|r}
\hline
  & paper & check & difference\\
\hline
attitude & 0.0602411 & 0.0515518 & 0.0086894\\
\hline
in & 0.0495368 & 0.0461179 & 0.0034189\\
\hline
contactOnly & 0.0975058 & 0.0451252 & 0.0523806\\
\hline
pgp\_donate\_end & 0.0203803 & 0.0405599 & -0.0201796\\
\hline
pgp\_amount\_end & -53.7397532 & -38.3723911 & -15.3673621\\
\hline
total (excludes pgp\_amount) & 0.2276640 & 0.1833548 & 0.0443092\\
\hline
\end{tabular}


\end{center}
\end{table}

This looks at heterogeneous effects by group (farmers and pastoralists).
This is a low-power analysis. There are no significant differences in
treatment effect by farmer/pastoralist.

Coefficients and p-values estimating farmer/pastoralist heterogeneous
effects were calculated with robust OLS regression using site-level
clusters and fixed effects for state. The regression interacted the
treatment indicator with the state indicator. Farmers were the reference
category so this table shows differences for pastoralists.

\begin{table}[H]
\begin{center}
\label{tab:farm_tab}
\caption{\textbf{Farmer-pastoralist differences in community-level analysis.} There are not significant differences between the effect of the contact intervention by farmer/pastoralist. The first column shows coefficients from OLS regression, the second column shows $p$-values from OLS regression.}
\smallskip

\begin{tabular}{l|r|r}
\hline
  & coefficient & p-value\\
\hline
Attitudes & -0.090 & 0.278\\
\hline
Perceptions of security & 0.021 & 0.905\\
\hline
Contact & 0.031 & 0.911\\
\hline
Percent Experiment & -0.014 & 0.913\\
\hline
Endorsement Experiment & -0.194 & 0.626\\
\hline
PGG donation & -0.064 & 0.386\\
\hline
PGG amount & 23.284 & 0.784\\
\hline
\end{tabular}


\end{center}
\end{table}

\begin{center}\rule{0.5\linewidth}{0.5pt}\end{center}

\hypertarget{appendix-h-survey-questions}{%
\subsection{Appendix H: Survey
Questions}\label{appendix-h-survey-questions}}

\textbf{Attitudes}

\begin{itemize}
\tightlist
\item
  With regards to someone from {[}X GROUP{]}, would you feel
  comfortable:

  \begin{itemize}
  \tightlist
  \item
    if they worked in your field?
  \item
    paying them to watch your animals?
  \item
    trading goods with them?
  \item
    sharing a meal with them?
  \item
    with a close relative marrying a person from {[}X GROUP{]}?
  \end{itemize}
\item
  From 1-5, how much do you trust people from {[}X GROUP{]} in your
  area?
\item
  Now I'm going to ask you questions about your community here in
  Benue/Nassarawa, including {[}X GROUP{]}. Please tell me how strongly
  you agree/disagree with each of the following statements: People in
  this area can be trusted.
\end{itemize}

\textbf{Contact}

\begin{itemize}
\tightlist
\item
  Now I'm going to ask you questions about your contact with {[}X
  GROUP{]} in your area.

  \begin{itemize}
  \tightlist
  \item
    Think of the market you go to most frequently. During the past
    month, have members of X GROUP gone to that market too? In the past
    month, how many times did you interact with X group in the market?
  \end{itemize}
\item
  In the past month, have you:

  \begin{itemize}
  \tightlist
  \item
    Joined a member of X group for a social event outside the home? How
    often?
  \item
    Hosted a member of X group for a ceremony in your home? How often?
  \item
    Gone to the home of a member of X group for a ceremony? How often?
  \item
    Have you interacted with members of X group in any other way in the
    past month?
  \end{itemize}
\end{itemize}

\textbf{Insecurity}

\begin{itemize}
\tightlist
\item
  In the last year were there any areas that you avoided going to or
  through because of insecurity during the night?
\item
  In the last year were there any areas that you avoided going to or
  through because of insecurity, during the day?
\item
  In the last year, did insecurity ever prevent you from:

  \begin{itemize}
  \tightlist
  \item
    Working when you wanted to work? About how many days were you unable
    to work?
  \item
    Going to the market?
  \item
    Getting water for the household?
  \item
    Going to your field/farm?
  \item
    Moving your animals to grazing areas?
  \item
    Moving your animals to water?
  \item
    Earning money or going to work?
  \item
    Going to school?
  \end{itemize}
\end{itemize}

\textbf{Endorsement Experiment}

\begin{itemize}
\tightlist
\item
  Imagine that there is a proposal by {[}\textbf{the Farmer's
  Cooperative Society}/\textbf{MACBAN}{]} for action to enhance access
  to clean water in rural areas. Though expensive, the proposal aims to
  bring fresh, clean water to hundreds of areas without access to it,
  including this one. If this were proposed, how would you feel about
  it?
\end{itemize}

\textbf{Percent Experiemnt}

\begin{itemize}
\tightlist
\item
  Think about groups that you might join in your leisure time. Would you
  join a group that had \textbf{5/25/50/75}\% X Group members?
\item
  Think about the community you live in. Would you live in a community
  that had \textbf{5/25/50/75}\% X Group members?
\end{itemize}

\textbf{Violence Placebo}

\begin{itemize}
\tightlist
\item
  Now I am going to ask you some questions about the use of violence. Is
  it always, sometimes, rarely, or never justified to use violence to do
  each of the following:

  \begin{itemize}
  \tightlist
  \item
    Retaliate against violence
  \item
    Defend one's group
  \item
    Maintain culture and traditions
  \item
    Defend one's religion
  \item
    Bring criminals to justice
  \item
    Force the government to change their policies
  \end{itemize}
\end{itemize}

\textbf{Public Goods Game}

``Thank you very much for participating in our survey. Before I go,
there is one last thing. As you may have heard, we have development
funds to use in this community. We have randomly selected you as one of
the 50 people to receive these funds. These funds are not for a Mercy
Corps project, but rather for you to keep personally or to donate to a
community fund.

We have 1,000 Naira to give to you. It is yours, and you can use it
either way--for yourself or for a community good.

Your community and {[}joint farmer/pastoralist community{]} have created
a project committee to whom you can donate this money so that it may be
used to help both communities. The project committee has 4 people from
each community. We have found a donor that will match the funds that you
all contribute to the project committee, so that if you donate 100 Naira
the project committee receives 300 Naira, and if you donate all 1,000
Naira the project committee receives 3,000 Naira. You are welcome to
donate none, some, or all of the money to the project committee.

These are your individual donation envelopes. All the donations will be
private -- only you will know how much money you donated. It essential
that you keep how much you give private -- please do not tell anyone. I
have with me a donation envelope to collect donations. Please go into
your home, put however much of the 1,000 Naira you wish to donate to the
project committee in the envelope, take whatever amount you want to keep
for yourself, and come back to place your envelope in the donation
envelope. Remember, you are welcome to donate none, some, or all of the
money to the project committee. After that we are finished and you may
continue your day. We will come back and publicly announce how much
money your community's project committee will receive.''

\begin{center}\rule{0.5\linewidth}{0.5pt}\end{center}

\hypertarget{appendix-i-alternative-explanations}{%
\subsection{Appendix I: Alternative
explanations}\label{appendix-i-alternative-explanations}}

It is possible that the effects of our intervention are due to impact of
the development project around which the contact was organized, rather
than intergroup contact itself. The effect could also be due to
mediation provided to some community leaders. We ran three analyses to
disentangle the effects of contact from the effects of the development
projects and mediation.

\hypertarget{effect-of-contact-vs.-effect-of-development-projects}{%
\subsubsection{Effect of contact vs.~effect of development
projects}\label{effect-of-contact-vs.-effect-of-development-projects}}

The first was to determine if treatment effects were significantly
larger for communities where larger proportions of people were aware of,
used, and perceived benefit from the projects. In treatment communities,
we predicted outcome change with (1) awareness of boreholes, (2) use of
boreholes, (3) awareness of quick-impact projects, and (4) perceived
benefit from quick-impact projects. If treatment effects are due to
development projects and not contact, we would expect larger effects
where a greater proportion of respondents were aware of, used, and
benefited from the development projects.

Our analysis shows that awareness, use, and benefit from the projects is
not significantly related to any outcome (the mean p-value is
\textasciitilde0.32). It's also not the case that any of those variables
separately (awareness, use, and benefit) are related to improvements on
any outcome. Though the p-values are insignificant, the mean t-statistic
for the awareness/use/benefit variables is positive, possibly suggesting
that the development projects, while not explaining the bulk of the
treatment effect, may have increased the effect. The tables show the
mean coefficient, \(p\)-value, and t-statistic for each outcome being
predicted by each ``benefit'' variable.

\begin{table}[H]
\begin{center}
\label{tab:benefit_df_tab}
\caption{\textbf{Benefit variables affect on outcomes.} Mean coefficient, $p$-value, and t-statistic for each outcome, across benefit variables. Benefit variables are not significantly related to any outcome.}
\smallskip

\begin{tabular}{l|r|r|r}
\hline
  & coefs & pvals & tstat\\
\hline
Attitudes & 0.1936077 & 0.1836764 & 0.9056779\\
\hline
Security & 0.2448961 & 0.2794338 & 1.2928731\\
\hline
Contact & 0.3207580 & 0.2643255 & 1.3698100\\
\hline
Contact Willingness & 0.3714763 & 0.0915421 & 1.9362679\\
\hline
Endorsement Exp & 0.1398422 & 0.6779022 & 0.3172168\\
\hline
PGG Donation & -0.0306856 & 0.2763725 & -1.3456686\\
\hline
PGG Amount & -35.9157762 & 0.4960119 & -0.7851551\\
\hline
All & NA & 0.3241806 & 0.5272889\\
\hline
\end{tabular}


\end{center}
\end{table}

\begin{table}[H]
\begin{center}
\label{tab:benefitVar_df_svy_tab}
\caption{\textbf{Benefit variables affect on outcomes.} Mean coefficient, $p$-value, and t-statistic for each benefit variables, across outcomes. This tables removes Public Goods Game outcomes because (1) their coefficients are on a different scale than the survey outcomes and (2) the coefficients and t-statistics are negative, which could hide the effect of benefit variables on the other outcomes. However, this table shows that benefit variables are not significantly related to any survey outcome.}
\smallskip

\begin{tabular}{l|r|r|r}
\hline
  & coefs & pvals & tstat\\
\hline
Borehole use & 0.3355362 & 0.2547888 & 1.5838113\\
\hline
Borehole aware & 0.2377543 & 0.3878788 & 0.9939178\\
\hline
QIP benefit & 0.0657282 & 0.3007221 & -0.0346520\\
\hline
QIP aware & 0.3774457 & 0.2541144 & 2.1143994\\
\hline
all & NA & 0.2993760 & 1.1643691\\
\hline
\end{tabular}


\end{center}
\end{table}

Second, we looked at whether pastoralists in Benue differed from the
rest of the sample. Pastoralists in Benue benefited least from the
development projects (especially the main project that built boreholes)
because they became displaced from where the boreholes were constructed
prior to the endline survey. Here we (1) confirm that pastoralists in
Benue were less likely to be aware of, use, or benefit from the
development projects and (2) see no significant outcome differences
between pastoralists in Benue and the rest of the sample (mean p-value
\textasciitilde0.58).

\begin{table}[H]
\begin{center}
\label{tab:BenPast_Benefit_tab}
\caption{\textbf{Benefit pastoralists awareness and benefit from development projects.} Coefficients, t-statistics, and $p$-values comparing Benue pastoralists in the treatment group to the rest of the treatment group. Benue pastoralists were significantly less likely to be aware of the boreholes or use the boreholes; they were marginally less likely to be aware of the quick-impact projects.}
\smallskip

\begin{tabular}{l|r|r|r}
\hline
  & estimate & statistic & p.value\\
\hline
Borehole Awareness & -0.4309759 & -4.3361033 & 0.0111897\\
\hline
Borehole Use & -0.3406948 & -5.1907843 & 0.0058327\\
\hline
QIP Awareness & -0.2907236 & -2.4706291 & 0.0662889\\
\hline
QIP Benefit & -0.0644359 & -0.6369732 & 0.5574075\\
\hline
\end{tabular}


\end{center}
\end{table}

\begin{table}[H]
\begin{center}
\label{tab:benPast_tab}
\caption{\textbf{Benefit outcomes for pastoralists in the treatment group.} Coefficients, $p$-values, and t-statistics comparing Benue pastoralists in the treatment group to the rest of the treatment group. Treatment effects are not smaller among Benue pastoralists than the rest of the treatment group. Note: the mean coefficient excludes the Public Goods Game, which is on a different scale than the other variables; it is included in the $p$-value and t-statistic.}
\smallskip

\begin{tabular}{l|r|r|r}
\hline
  & coef & pvalue & tstat\\
\hline
Attitudes & -0.0317303 & 0.8260431 & -0.2398319\\
\hline
Security & -0.1668610 & 0.6400484 & -0.5187857\\
\hline
Contact & 0.2343247 & 0.3188031 & 1.1945197\\
\hline
Contact Willingness & 0.0301262 & 0.8672572 & 0.1820363\\
\hline
Endorsement Exp & -0.5074066 & 0.1968861 & -1.6571511\\
\hline
PGG Donation & -0.0342750 & 0.5305509 & -0.7077525\\
\hline
PGG Amount & 49.9043759 & 0.6549041 & 0.4949424\\
\hline
All & -0.0883094 & 0.5763561 & -0.1788604\\
\hline
\end{tabular}


\end{center}
\end{table}

\hypertarget{effect-of-contact-vs.-effect-of-mediation}{%
\subsubsection{Effect of contact vs.~effect of
mediation}\label{effect-of-contact-vs.-effect-of-mediation}}

The third analysis concerned mediation. While we cannot rule out that
the effect was due to the meditation training, only 52 of the over 1000
endline respondents in treatment sites were aware of the mediation
intervention.

\begin{table}[H]
\begin{center}
\label{tab:mediat_tab}
\caption{\textbf{Mediation Exposure.} Exposure to mediation in treatment and control sites in endline survey. Only 52 randomly selected respondents in treatment had any exposure to mediation. No control respondents had exposure to mediation.}
\smallskip

\begin{tabular}{l|r|r}
\hline
  & Control & Treatment\\
\hline
None & 504 & 996\\
\hline
Exposure & 0 & 52\\
\hline
\end{tabular}


\end{center}
\end{table}

\begin{center}\rule{0.5\linewidth}{0.5pt}\end{center}

\hypertarget{appendix-j-family-wise-error-rate-fwer}{%
\subsection{Appendix J: Family-Wise Error Rate
(FWER)}\label{appendix-j-family-wise-error-rate-fwer}}

To control the FWER, we conducted within-family hypothesis corrections
using the Holm correction. For two of our outcomes (pastoralists in
market and perceptions of insecurity), the findings remain statistically
significant. For the two other outcomes (self-reported contact and
attitudes), significance levels shift from statistically significant to
marginally significant (.04 to .08). This means we have a significant
\(p\)-value for our hypothesis families about intergroup contact and
about insecurity; we have a marginally significant \(p\)-value for our
hypothesis family about attitudes; we have no signifcant \(p\)-value for
our hypothesis family about cooperation.

Note: the outcome measuring pastoralists in the market remains
statistically significant even if using the bonferonni correction to
correct for all (vs.~within family) hypothesis tests in the analysis.

The exploratory individual-level data has only one outcome per family
(except for the Cooperation family, for which no \(p\)-values were
significant without correction). We therefore do not show ``corrected''
individual-level \(p\)-values. If we put all individual-level hypotheses
into a single family and use the Holm correction, the
\emph{Self-reported contact} outcome changes from significant
(\(p\)=0.018) to marginally significant(\(p\)=0.088). None of the other
outcomes were significant or marginally significant before correction.

\begin{table}[H]
\begin{center}
\label{tab:holm_tab}
\caption{\textbf{P-values controlling the family-wise error rate.}}
\smallskip

\begin{tabular}{l|r|r|l}
\hline
Outcome & None & Holm & Family\\
\hline
Pastoralists in Market & 0.0014 & 0.0070 & Contact\\
\hline
Self-reported contact & 0.0595 & 0.2380 & Contact\\
\hline
Contact willingness & 0.2389 & 0.5568 & Contact\\
\hline
Farmers in Market & 0.2416 & 0.5568 & Contact\\
\hline
Outgroup event attendance & 0.1856 & 0.5568 & Contact\\
\hline
Perceptions of security & 0.0205 & 0.0205 & Insecurity\\
\hline
Self-reported attitudes & 0.0448 & 0.0896 & Attitudes\\
\hline
Endorsement Experiment & 0.2117 & 0.2117 & Attitudes\\
\hline
Public Goods Donation & 0.2938 & 0.5876 & Cooperation\\
\hline
Public Goods Amount & 0.8522 & 0.8522 & Cooperation\\
\hline
\end{tabular}


\end{center}
\end{table}

\begin{center}\rule{0.5\linewidth}{0.5pt}\end{center}

\hypertarget{appendix-k-power-analysis}{%
\subsection{Appendix K: Power
Analysis}\label{appendix-k-power-analysis}}

At community level, we could detect effects of \textasciitilde0.60 SDs
with 0.80 power. We expected better power for our actual analysis
because our power analysis does not use randomization inference to
generate true p-values. We simulated our community-level power analysis
with the following code.

\begin{Shaded}
\begin{Highlighting}[]
\CommentTok{\# using insecurity as the default outcome, as it is our strongest survey outcome at the community level. outcome\_list\_qip[2] = "in\_cw".}
\NormalTok{bigPow.fn }\OtherTok{\textless{}{-}}\ControlFlowTok{function}\NormalTok{(nsims, }\AttributeTok{var=}\NormalTok{outcome\_list\_qip[}\DecValTok{2}\NormalTok{], tau)}
\NormalTok{\{}
\NormalTok{  newPow.fn }\OtherTok{\textless{}{-}} \ControlFlowTok{function}\NormalTok{(var, tau)}
\NormalTok{\{}
    \CommentTok{\# 6 TR sites from Nas, 4 from Ben}
\NormalTok{    newtr\_nas }\OtherTok{\textless{}{-}} \FunctionTok{sample}\NormalTok{(}\FunctionTok{unique}\NormalTok{(ag.df}\SpecialCharTok{$}\NormalTok{psu[ag.df}\SpecialCharTok{$}\NormalTok{state }\SpecialCharTok{\%in\%} \StringTok{"nas"}\NormalTok{]), }\AttributeTok{size=}\DecValTok{6}\NormalTok{)}
\NormalTok{    newtr\_ben }\OtherTok{\textless{}{-}} \FunctionTok{sample}\NormalTok{(}\FunctionTok{unique}\NormalTok{(ag.df}\SpecialCharTok{$}\NormalTok{psu[ag.df}\SpecialCharTok{$}\NormalTok{state }\SpecialCharTok{\%in\%} \StringTok{"ben"}\NormalTok{]), }\AttributeTok{size=}\DecValTok{4}\NormalTok{)}
\NormalTok{    newtr }\OtherTok{\textless{}{-}} \FunctionTok{c}\NormalTok{(}\FunctionTok{as.character}\NormalTok{(newtr\_nas), }\FunctionTok{as.character}\NormalTok{(newtr\_ben))}
\NormalTok{    df }\OtherTok{\textless{}{-}}\NormalTok{ ag.df}
\NormalTok{    df[,}\StringTok{"newtr"}\NormalTok{] }\OtherTok{\textless{}{-}} \FunctionTok{ifelse}\NormalTok{(df}\SpecialCharTok{$}\NormalTok{psu }\SpecialCharTok{\%in\%}\NormalTok{ newtr, }\DecValTok{1}\NormalTok{, }\DecValTok{0}\NormalTok{)}
  
  \CommentTok{\# make endline outcome with TR effect tau}
\NormalTok{  df[, }\FunctionTok{paste0}\NormalTok{(var,}\StringTok{"\_end"}\NormalTok{)] }\OtherTok{\textless{}{-}}\NormalTok{ (df[,}\FunctionTok{paste0}\NormalTok{(var,}\StringTok{"\_end"}\NormalTok{)]}\SpecialCharTok{{-}}\FunctionTok{mean}\NormalTok{(df[,}\FunctionTok{paste0}\NormalTok{(var,}\StringTok{"\_end"}\NormalTok{)]))}\SpecialCharTok{/}\FunctionTok{sd}\NormalTok{(df[,}\FunctionTok{paste0}\NormalTok{(var,}\StringTok{"\_end"}\NormalTok{)])}
  \CommentTok{\#scale(df[,paste0(var,"\_end")])}
\NormalTok{  df[df}\SpecialCharTok{$}\NormalTok{newtr }\SpecialCharTok{\%in\%} \DecValTok{1}\NormalTok{, }\FunctionTok{paste0}\NormalTok{(var,}\StringTok{"\_end"}\NormalTok{)] }\OtherTok{\textless{}{-}}\NormalTok{ df[df}\SpecialCharTok{$}\NormalTok{newtr }\SpecialCharTok{\%in\%} \DecValTok{1}\NormalTok{, }\FunctionTok{paste0}\NormalTok{(var,}\StringTok{"\_end"}\NormalTok{)]}\SpecialCharTok{+}\NormalTok{tau}
  
  \CommentTok{\# for baseline control, also scale}
\NormalTok{  df[, }\FunctionTok{paste0}\NormalTok{(var,}\StringTok{"\_base"}\NormalTok{)] }\OtherTok{\textless{}{-}}\NormalTok{ (df[,}\FunctionTok{paste0}\NormalTok{(var,}\StringTok{"\_base"}\NormalTok{)]}\SpecialCharTok{{-}}\FunctionTok{mean}\NormalTok{(df[,}\FunctionTok{paste0}\NormalTok{(var,}\StringTok{"\_base"}\NormalTok{)]))}\SpecialCharTok{/}\FunctionTok{sd}\NormalTok{(df[,}\FunctionTok{paste0}\NormalTok{(var,}\StringTok{"\_base"}\NormalTok{)])}
  
  \CommentTok{\# lm}
\NormalTok{  lm1 }\OtherTok{\textless{}{-}} \FunctionTok{lm\_robust}\NormalTok{(df[,}\FunctionTok{paste0}\NormalTok{(var,}\StringTok{"\_end"}\NormalTok{)]}\SpecialCharTok{\textasciitilde{}}\NormalTok{df[,}\StringTok{\textquotesingle{}newtr\textquotesingle{}}\NormalTok{]}\SpecialCharTok{+}\NormalTok{df[,}\FunctionTok{paste0}\NormalTok{(var,}\StringTok{"\_base"}\NormalTok{)]}\SpecialCharTok{+}\NormalTok{state, }
                   \AttributeTok{clusters =}\NormalTok{ psu, }\AttributeTok{data=}\NormalTok{df)}
  
\NormalTok{  want }\OtherTok{\textless{}{-}} \FunctionTok{tidy}\NormalTok{(lm1)[}\DecValTok{2}\NormalTok{,}\DecValTok{5}\NormalTok{]}
  \FunctionTok{return}\NormalTok{(want)}
\NormalTok{  \}}
  
\NormalTok{  check }\OtherTok{\textless{}{-}} \FunctionTok{do}\NormalTok{(nsims)}\SpecialCharTok{*}\FunctionTok{newPow.fn}\NormalTok{(}\AttributeTok{var=}\NormalTok{var, }\AttributeTok{tau=}\NormalTok{tau)}
\NormalTok{  pval }\OtherTok{\textless{}{-}} \FunctionTok{mean}\NormalTok{(check}\SpecialCharTok{\textless{}}\FloatTok{0.05}\NormalTok{)}
  \FunctionTok{return}\NormalTok{(pval)}
\NormalTok{\}}

\CommentTok{\# run power analysis for tau 0{-}1}
\NormalTok{possibleTaus }\OtherTok{\textless{}{-}} \FunctionTok{seq}\NormalTok{(}\DecValTok{0}\NormalTok{,}\DecValTok{1}\NormalTok{,}\FloatTok{0.1}\NormalTok{)}
\NormalTok{possibleTaus }\OtherTok{\textless{}{-}} \FunctionTok{as.data.frame}\NormalTok{(possibleTaus)}
\FunctionTok{system.time}\NormalTok{(}
\ControlFlowTok{for}\NormalTok{(i }\ControlFlowTok{in} \DecValTok{1}\SpecialCharTok{:}\FunctionTok{nrow}\NormalTok{(possibleTaus))}
\NormalTok{\{}
\NormalTok{  possibleTaus[i, }\StringTok{"pow"}\NormalTok{] }\OtherTok{\textless{}{-}} \FunctionTok{bigPow.fn}\NormalTok{(}\AttributeTok{nsims=}\DecValTok{3000}\NormalTok{, }\AttributeTok{tau=}\NormalTok{possibleTaus[i,}\DecValTok{1}\NormalTok{])}
\NormalTok{\}}
\NormalTok{)}
\NormalTok{possibleTaus}
\end{Highlighting}
\end{Shaded}

\begin{figure}[H]
\centering
\includegraphics[width=\linewidth]{../data_and_code/review/power_figure.png}
\caption{Community-level power analysis.}\label{fig:pow_comm}
\end{figure}

At individual level, we could detect effects of \textasciitilde0.40 SDs
with 0.80 power. We expected better power for our actual analysis
because our power analysis does not use randomization inference to
generate true p-values. We used the
\href{https://egap.shinyapps.io/power-app/}{EGAP power calculator} for
the individual-level power analysis. The parameters were: Alpha = 0.05,
Tau = 0.40 SD, ICC=0.05, 15 clusters per arm, up to 300 respondents per
cluster.

\begin{figure}[H]
\centering
\includegraphics[width=\linewidth]{../data_and_code/review/power_figure_ind.png}
\caption{Individual-level power analysis.}\label{fig:pow_ind}
\end{figure}

\begin{center}\rule{0.5\linewidth}{0.5pt}\end{center}

\hypertarget{appendix-l-difference-in-differences-plots}{%
\subsection{Appendix L: Difference-in-differences
plots}\label{appendix-l-difference-in-differences-plots}}

Many effects are driven by worsening conditions in control communities.
Given the small number of control communities, it's possible that the
effects could be driven by a particularly bad period in a single control
community. It's also possible that there could be ceiling or floor
effects. These plots show that (1) the effects are not driven by a large
drop in a single control community/site and (2) the overall effect is
not demonstrating a floor below which no community-site falls.

\hypertarget{survey-outcomes-1}{%
\subsubsection{Survey outcomes}\label{survey-outcomes-1}}

\begin{figure}[H]
\centering
\includegraphics[width=\linewidth]{../data_and_code/survey_dat/figs/did_plots/attitudeComm_plot.png}
\caption{Difference-in-differences plot comparing change in treatment to change in control.}\label{fig:att_comm}
\end{figure}

\begin{figure}[H]
\centering
\includegraphics[width=\linewidth]{../data_and_code/survey_dat/figs/did_plots/attitudeComm.plot_disag.png}
\caption{Each point represents a community. This graph shows (1) that overall changes are not driven by a large change in a single community and (2) that the overall change does not reflect a ceiling or floor effect.}\label{fig:att_comm_dis}
\end{figure}

\begin{figure}[H]
\centering
\includegraphics[width=\linewidth]{../data_and_code/survey_dat/figs/did_plots/finished_didPlots/conComm_plot.png}
\caption{Difference-in-differences plot comparing change in treatment to change in control.}\label{fig:con_comm}
\end{figure}

\begin{figure}[H]
\centering
\includegraphics[width=\linewidth]{../data_and_code/survey_dat/figs/did_plots/conComm_plot_disag.png}
\caption{Each point represents a community. This graph shows (1) that overall changes are not driven by a large change in a single community and (2) that the overall change does not reflect a ceiling or floor effect.}\label{fig:con_comm_dis}
\end{figure}

\begin{figure}[H]
\centering
\includegraphics[width=\linewidth]{../data_and_code/survey_dat/figs/did_plots/finished_didPlots/inComm_plot.png}
\caption{Difference-in-differences plot comparing change in treatment to change in control.}\label{fig:in_comm}
\end{figure}

\begin{figure}[H]
\centering
\includegraphics[width=\linewidth]{../data_and_code/survey_dat/figs/did_plots/inComm_plot_disag.png}
\caption{Each point represents a community. This graph shows (1) that overall changes are not driven by a large change in a single community and (2) that the overall change does not reflect a ceiling or floor effect.}\label{fig:in_comm_dis}
\end{figure}

\begin{figure}[H]
\centering
\includegraphics[width=\linewidth]{../data_and_code/survey_dat/figs/did_plots/finished_didPlots/endComm_plot.png}
\caption{Difference-in-differences plot comparing change in treatment to change in control.}\label{fig:end_comm}
\end{figure}

\begin{figure}[H]
\centering
\includegraphics[width=\linewidth]{../data_and_code/survey_dat/figs/did_plots/finished_didPlots/randComm_plot.png}
\caption{Difference-in-differences plot comparing change in treatment to change in control.}\label{fig:perc_comm}
\end{figure}

\hypertarget{behavioral-outcomes}{%
\subsubsection{Behavioral outcomes}\label{behavioral-outcomes}}

\begin{figure}[H]
\centering
\includegraphics[width=\linewidth]{../data_and_code/obs_dat/b_analysis/market_pasts_TrTime.plot.png}
\caption{Difference-in-differences plot comparing change in treatment to change in control.}\label{fig:market_past_TrTime}
\end{figure}

\begin{figure}[H]
\centering
\includegraphics[width=\linewidth]{../data_and_code/obs_dat/b_analysis/market_pasts_siteTime.plot.png}
\caption{Each point represents a site. This graph shows (1) that overall changes are not driven by a large change in a single community and (2) that the overall change does not reflect a ceiling or floor effect.}\label{fig:market_past_siteTime}
\end{figure}

\begin{figure}[H]
\centering
\includegraphics[width=\linewidth]{../data_and_code/obs_dat/b_analysis/market_farms_TrTime.plot.png}
\caption{Difference-in-differences plot comparing change in treatment to change in control.}\label{fig:market_farm_TrTime}
\end{figure}

\begin{figure}[H]
\centering
\includegraphics[width=\linewidth]{../data_and_code/obs_dat/b_analysis/market_farms_siteTime.plot.png}
\caption{Each point represents a site. This graph shows (1) that overall changes are not driven by a large change in a single community and (2) that the overall change does not reflect a ceiling or floor effect.}\label{fig:market_farm_siteTime}
\end{figure}

\begin{figure}[H]
\centering
\includegraphics[width=\linewidth]{../data_and_code/obs_dat/b_analysis/event_outgroup_TrTime.plot.png}
\caption{Difference-in-differences plot comparing change in treatment to change in control.}\label{fig:market_past_TrTime}
\end{figure}

\begin{figure}[H]
\centering
\includegraphics[width=\linewidth]{../data_and_code/obs_dat/b_analysis/events_outgroup_siteTime.plot.png}
\caption{Each point represents a site. This graph shows (1) that overall changes are not driven by a large change in a single community and (2) that the overall change does not reflect a ceiling or floor effect.}\label{fig:events_outgroup_siteTime}
\end{figure}

\begin{center}\rule{0.5\linewidth}{0.5pt}\end{center}

\hypertarget{appendix-m-intervention-and-site-selection}{%
\subsection{Appendix M: Intervention and
site-selection}\label{appendix-m-intervention-and-site-selection}}

\hypertarget{intervention-details}{%
\subsubsection{Intervention Details}\label{intervention-details}}

To address the rising conflict in the Middle Belt in Nigeria, Mercy
Corps implemented a four-year, USAID-funded program titled Engaging
Communities for Peace in Nigeria, targeted at over 9,000 people at 10
Middle Belt sites (where a ``site'' is a location that contains two
conflicting communities). The main objective of the program was to
foster positive contact between farmers and pastoralists, with the aim
of improving their attitudes toward and relationships with each other.
The program included three interventions.

The first intervention involved creating joint project committees in
which farmers and pastoralists worked together to design and implement
development projects that would address intergroup tensions. The process
for selecting projects was similar to that of many community-driven
development programs. It started with a series of community meetings,
beginning with separate farmer and pastoralist meetings that built up to
joint decision-making meetings with the two groups together on the
project committees. Project committees met weekly during the dry season
and bi-weekly during farming and harvest season. Each joint project
committee included an even number of farmers and pastoralists, as well
as women and youth representatives, and totaled between 12 and 15
members.

Each committee received two grants, one for quick-impact projects, of
approximately \$2,000, and one for joint economic projects, of
approximately \$25,000. To inform the selection of projects, the project
committees conducted a participatory needs assessment to gather the
opinions of various demographic groups within their communities. The
project committees then used this information, along with a conflict and
resource mapping exercise, to identify resource-based drivers of
conflict and select projects to implement with the grant money. The
quick-impact projects were conceived as a trust-building initiative,
intended to demonstrate that cooperation was possible and in the
interest of both groups. Quick-impact projects, managed by both farmers
and pastoralists, included hand pumps, construction or rehabilitation of
market stalls and schools, rehabilitation of health centers, and
construction of fences along grazing routes to protect farmlands. The
joint economic projects aimed to address an underlying issue related to
the conflict: sharing of scarce resources that impact livelihoods.
Pollution of water, affecting both farming and livestock, was the
primary issue people raised. As a result, each project committee decided
to build a new borehole well, with the committee managing the project
and with farmer and pastoralist youth helping to construct the wells.
For most communities, the borehole was erected near the end of the 18
month program.

The second intervention was training community leaders in how to mediate
disputes so that conflict did not escalate into violence. To help
alleviate violence, two joint committees were formed: peace and early
warning / early response. In some venues, a peace committee was already
in place. In those instances, we worked with the preexisting peace
committee to ensure that it was balanced in terms of farmers and
pastoralists, as well as gender. Over the course of the two years, 120
people were trained in mediation, and they went on to resolve 528
disputes around local grazing routes, seasonal access to water points,
crop damage, cutting down of trees, and water pollution by animals. That
said, only 52 of over 1000 respondents in the treatment group had
exposure to mediation at endline.

The third intervention was conflict prevention forums for the larger
community, in which farmers and pastoralists came together to discuss
issues and policies that were affecting them. Government officials also
attended these events. Due to time, this aspect of the program was only
lightly implemented.

\hypertarget{site-selection-details}{%
\subsubsection{Site Selection Details}\label{site-selection-details}}

The scoping exercise initially identified more than thirty potential
implementation sites with a history of violence. The ECPN implementation
team visited these sites to establish community need and obtain
community consent for potentially becoming part of the ECPN program.
From these visits we narrowed down the list of implementation sites to
24. These 24 sites received a preliminary survey of 10 individuals per
farmer and pastoralist community to further identify need. This survey
revealed one site that was too close geographically to a larger site,
one site that was too remote for feasible implementation, and four sites
did not fulfill our ``demonstrated need'' criteria. Of the remaining 18,
three were lost before random assignment or implementation of the
program. From the fifteen remaining sites, we randomly assigned ten to
intervention, where ECPN would be implemented, and five to control,
where ECPN would not be implemented.

\end{document}
