% Options for packages loaded elsewhere
\PassOptionsToPackage{unicode}{hyperref}
\PassOptionsToPackage{hyphens}{url}
%
\documentclass[
]{article}
\usepackage{amsmath,amssymb}
\usepackage{lmodern}
\usepackage{float}
\usepackage{iftex}
\ifPDFTeX
  \usepackage[T1]{fontenc}
  \usepackage[utf8]{inputenc}
  \usepackage{textcomp} % provide euro and other symbols
\else % if luatex or xetex
  \usepackage{unicode-math}
  \defaultfontfeatures{Scale=MatchLowercase}
  \defaultfontfeatures[\rmfamily]{Ligatures=TeX,Scale=1}
\fi
% Use upquote if available, for straight quotes in verbatim environments
\IfFileExists{upquote.sty}{\usepackage{upquote}}{}
\IfFileExists{microtype.sty}{% use microtype if available
  \usepackage[]{microtype}
  \UseMicrotypeSet[protrusion]{basicmath} % disable protrusion for tt fonts
}{}
\makeatletter
\@ifundefined{KOMAClassName}{% if non-KOMA class
  \IfFileExists{parskip.sty}{%
    \usepackage{parskip}
  }{% else
    \setlength{\parindent}{0pt}
    \setlength{\parskip}{6pt plus 2pt minus 1pt}}
}{% if KOMA class
  \KOMAoptions{parskip=half}}
\makeatother
\usepackage{xcolor}
\usepackage[margin=1in]{geometry}
\usepackage{color}
\usepackage{fancyvrb}
\newcommand{\VerbBar}{|}
\newcommand{\VERB}{\Verb[commandchars=\\\{\}]}
\DefineVerbatimEnvironment{Highlighting}{Verbatim}{commandchars=\\\{\}}
% Add ',fontsize=\small' for more characters per line
\usepackage{framed}
\definecolor{shadecolor}{RGB}{248,248,248}
\newenvironment{Shaded}{\begin{snugshade}}{\end{snugshade}}
\newcommand{\AlertTok}[1]{\textcolor[rgb]{0.94,0.16,0.16}{#1}}
\newcommand{\AnnotationTok}[1]{\textcolor[rgb]{0.56,0.35,0.01}{\textbf{\textit{#1}}}}
\newcommand{\AttributeTok}[1]{\textcolor[rgb]{0.77,0.63,0.00}{#1}}
\newcommand{\BaseNTok}[1]{\textcolor[rgb]{0.00,0.00,0.81}{#1}}
\newcommand{\BuiltInTok}[1]{#1}
\newcommand{\CharTok}[1]{\textcolor[rgb]{0.31,0.60,0.02}{#1}}
\newcommand{\CommentTok}[1]{\textcolor[rgb]{0.56,0.35,0.01}{\textit{#1}}}
\newcommand{\CommentVarTok}[1]{\textcolor[rgb]{0.56,0.35,0.01}{\textbf{\textit{#1}}}}
\newcommand{\ConstantTok}[1]{\textcolor[rgb]{0.00,0.00,0.00}{#1}}
\newcommand{\ControlFlowTok}[1]{\textcolor[rgb]{0.13,0.29,0.53}{\textbf{#1}}}
\newcommand{\DataTypeTok}[1]{\textcolor[rgb]{0.13,0.29,0.53}{#1}}
\newcommand{\DecValTok}[1]{\textcolor[rgb]{0.00,0.00,0.81}{#1}}
\newcommand{\DocumentationTok}[1]{\textcolor[rgb]{0.56,0.35,0.01}{\textbf{\textit{#1}}}}
\newcommand{\ErrorTok}[1]{\textcolor[rgb]{0.64,0.00,0.00}{\textbf{#1}}}
\newcommand{\ExtensionTok}[1]{#1}
\newcommand{\FloatTok}[1]{\textcolor[rgb]{0.00,0.00,0.81}{#1}}
\newcommand{\FunctionTok}[1]{\textcolor[rgb]{0.00,0.00,0.00}{#1}}
\newcommand{\ImportTok}[1]{#1}
\newcommand{\InformationTok}[1]{\textcolor[rgb]{0.56,0.35,0.01}{\textbf{\textit{#1}}}}
\newcommand{\KeywordTok}[1]{\textcolor[rgb]{0.13,0.29,0.53}{\textbf{#1}}}
\newcommand{\NormalTok}[1]{#1}
\newcommand{\OperatorTok}[1]{\textcolor[rgb]{0.81,0.36,0.00}{\textbf{#1}}}
\newcommand{\OtherTok}[1]{\textcolor[rgb]{0.56,0.35,0.01}{#1}}
\newcommand{\PreprocessorTok}[1]{\textcolor[rgb]{0.56,0.35,0.01}{\textit{#1}}}
\newcommand{\RegionMarkerTok}[1]{#1}
\newcommand{\SpecialCharTok}[1]{\textcolor[rgb]{0.00,0.00,0.00}{#1}}
\newcommand{\SpecialStringTok}[1]{\textcolor[rgb]{0.31,0.60,0.02}{#1}}
\newcommand{\StringTok}[1]{\textcolor[rgb]{0.31,0.60,0.02}{#1}}
\newcommand{\VariableTok}[1]{\textcolor[rgb]{0.00,0.00,0.00}{#1}}
\newcommand{\VerbatimStringTok}[1]{\textcolor[rgb]{0.31,0.60,0.02}{#1}}
\newcommand{\WarningTok}[1]{\textcolor[rgb]{0.56,0.35,0.01}{\textbf{\textit{#1}}}}
\usepackage{graphicx}
\makeatletter
\def\maxwidth{\ifdim\Gin@nat@width>\linewidth\linewidth\else\Gin@nat@width\fi}
\def\maxheight{\ifdim\Gin@nat@height>\textheight\textheight\else\Gin@nat@height\fi}
\makeatother
% Scale images if necessary, so that they will not overflow the page
% margins by default, and it is still possible to overwrite the defaults
% using explicit options in \includegraphics[width, height, ...]{}
\setkeys{Gin}{width=\maxwidth,height=\maxheight,keepaspectratio}
% Set default figure placement to htbp
\makeatletter
\def\fps@figure{htbp}
\makeatother
\setlength{\emergencystretch}{3em} % prevent overfull lines
\providecommand{\tightlist}{%
  \setlength{\itemsep}{0pt}\setlength{\parskip}{0pt}}
\setcounter{secnumdepth}{-\maxdimen} % remove section numbering
\ifLuaTeX
  \usepackage{selnolig}  % disable illegal ligatures
\fi
\IfFileExists{bookmark.sty}{\usepackage{bookmark}}{\usepackage{hyperref}}
\IfFileExists{xurl.sty}{\usepackage{xurl}}{} % add URL line breaks if available
\urlstyle{same} % disable monospaced font for URLs
\hypersetup{
  pdftitle={ecpn\_appendix},
  hidelinks,
  pdfcreator={LaTeX via pandoc}}

\title{ecpn\_appendix}
\author{}
\date{\vspace{-2.5em}}

\begin{document}
\maketitle

\hypertarget{appendices}{%
\section{Appendices}\label{appendices}}

Desired appendices.

\begin{itemize}
\tightlist
\item
  Explaining randomization inference and bootstrapping
\item
  Robustness checks for analysis

  \begin{itemize}
  \tightlist
  \item
    Additive indices instead of ICW indices
  \item
    Results with differencing method vs results with controlling-for
    method
  \item
    raw vs ranked variables to create the index
  \end{itemize}
\item
  Placebo analysis + components of placebo
\item
  Balance:

  \begin{itemize}
  \tightlist
  \item
    Communities/site: TR vs CO within state \& overall for all baseline
    outcomes (Separate for svy and obs)
  \item
    Individuals: Direct/indirect/control == unneeded because exploratory
    \& no assumption of balance.
  \end{itemize}
\item
  Panel analysis: big table of results (lower priority)

  \begin{itemize}
  \tightlist
  \item
    Additive indices instead of ICW indices
  \item
    Results with differencing method vs results with controlling-for
    method
  \item
    raw vs ranked variables to create the index
  \end{itemize}
\item
  State-level differences \& farmer-pastoralist differences (ready for
  reviewers)
\item
  Survey questions
\end{itemize}

\hypertarget{appendix-a-randomization-inference-and-bootstrapping}{%
\subsection{Appendix A: Randomization Inference and
Bootstrapping}\label{appendix-a-randomization-inference-and-bootstrapping}}

Randomization inference and bootstrapping are nonparametric methods to
generate \(p\)-values (randomization inference) and confidence intervals
(bootstrapping). With \emph{randomization inference}, we first shuffle
the treatment variable to break the relationship between treatment and
outcomes. Next we regress outcomes on treatment using our regression
equation and store the resulting coefficient. Lastly, we repeat that
process 10,000 times to create the distribution of coefficients we would
observe if treatment had no effect on outcomes -- the null hypothesis.
Our \(p\)-value is the proportion of the null distribution that is
greater than or equal to our observed coefficient.

\emph{Bootstrapping} for standard errors is similar, but instead of
shuffling the treatment indicator we resample units with replacement. By
resampling with replacement, we create the empirical distribution of our
data and the range of possible treatment effects we might observe if we
repeated the experiment 10,000 times. The treatment effect at the 2.5th
percentile and at the 97.5th percentile are equivalent to a 95\%
confidence interval {[}@efron1994introduction{]}.

In each of these procedures, we mimic our randomization process by
randomizing/resampling the intervention to communities in site-level
clusters and within state blocks. This means that both communities in an
implementation site (farmers and pastoralists) will always be
treated/sampled together and that assignment to the intervention and
resampling are conducted separately in Nassarawa and Benue, just as the
intervention was assigned in this study. This procedure ensures that our
null distribution (for \(p\)-values) is created by randomizing the
intervention between exchangeable units and that our empirical
distribution (for confidence intervals) is created by resampling units
as they were sampled.

\hypertarget{appendix-b-robustness-checks-for-analysis}{%
\subsection{Appendix B: Robustness checks for
analysis}\label{appendix-b-robustness-checks-for-analysis}}

These tables shows results with different ways of making indices
(additive vs inverse-covariance weighted), different models for
estimating effects (differencing vs controlling-for), and different ways
of coding count variables (raw vs ranked). Each table is an outcome.
Rows are results for different ways of creating the outcomes. Columns
show the coefficient from OLS regression, true p-value from
randomization inference, and a binary ``base'' indicator showing which
method was used in the paper.

index: inverse-covariance weights (``cw'') or additive (``ind''). model:
controlling-for (``cont'') or differencing (``diff''). Note that not not
all outcomes are indices, some outcomes are endline only, and only
contact outcomes use count variables.

\begin{verbatim}
## [[1]]
##                                 coef  truep  version base
## attitude_cw~treatment     0.11601349 0.0443  cw_cont    1
## attitude_index~treatment  0.09280439 0.0378 ind_cont    0
## attitude_cw~treatment1    0.10026696 0.1421  cw_diff    0
## attitude_index~treatment1 0.07306274 0.1218 ind_diff    0
## 
## [[2]]
##                            coef  truep  version base
## in_cw~treatment      0.09790809 0.0304  cw_cont    0
## in_index~treatment  -0.01045169 0.5934 ind_cont    0
## in_cw~treatment1     0.15909111 0.0229  cw_diff    1
## in_index~treatment1  0.05375752 0.2068 ind_diff    0
## 
## [[3]]
##                                     coef  truep  version base
## contactOnly_cw~treatment     0.012937275 0.4182  cw_cont    0
## contactOnly_index~treatment  0.003357323 0.4207 ind_cont    0
## contactOnly_cw~treatment1    0.137779036 0.0624  cw_diff    1
## contactOnly_index~treatment1 0.014713596 0.1825 ind_diff    0
## 
## [[4]]
##                        coef  truep  version base
## rMean~treatment  0.09327385 0.0491  cw_cont    0
## 2                        NA     NA ind_cont    0
## rMean~treatment1 0.06213299 0.2420  cw_diff    1
## 4                        NA     NA ind_diff    0
## 
## [[5]]
##                         coef  truep  version base
## end_exp~treatment  0.1225162 0.2161  cw_cont    0
## 2                         NA     NA ind_cont    0
## end_exp~treatment1 0.1225162 0.2098  cw_diff    1
## 4                         NA     NA ind_diff    0
## 
## [[6]]
##                                coef truep  version base
## pgp_donate_end~treatment 0.02241997 0.284  cw_cont   NA
## 2                                NA    NA ind_cont   NA
## 3                                NA    NA  cw_diff   NA
## 4                                NA    NA ind_diff   NA
## 
## [[7]]
##                              coef  truep  version base
## pgp_amount_end~treatment -35.1235 0.8435  cw_cont   NA
## 2                              NA     NA ind_cont   NA
## 3                              NA     NA  cw_diff   NA
## 4                              NA     NA ind_diff   NA
\end{verbatim}

\begin{verbatim}
##                                       coef  truep
## contactOnly_cw~treatment        0.01293728 0.4204
## contactOnly_cats_cw~treatment   0.01695302 0.3772
## contactOnly_raw_cw~treatment   -0.01988471 0.6021
## contactOnly_cw~treatment1       0.13777904 0.0653
## contactOnly_cats_cw~treatment1  0.12048417 0.0640
## contactOnly_raw_cw~treatment1   0.07120127 0.2072
\end{verbatim}

\begin{table}[H]
\begin{center}

\begin{tabular}{l|r|r|r}
\hline
  & coefficient & p-value & base\\
\hline
Controlling-for \& ICW & 0.116 & 0.044 & 1\\
\hline
Controlling-for \& Additive & 0.093 & 0.038 & 0\\
\hline
Differencing \& ICW & 0.100 & 0.142 & 0\\
\hline
Differencing \& Additive & 0.073 & 0.122 & 0\\
\hline
\end{tabular}


\caption{\label{tab:attitude_tab}\textbf{Effect of ECPN on main outcomes with alternative methods of estimation and index construction.} The first column shows coefficients from OLS regression, the second column shows $p$-values from randomization inference, and the third column shows which method was used in the paper.}
\end{center}
\end{table}

\hypertarget{appendix-b-results-with-additive-indices}{%
\subsection{Appendix B: Results with Additive
Indices}\label{appendix-b-results-with-additive-indices}}

These tables show results for self-report survey outcomes made with
additive indices. The tables include the coefficients and \(p\)-values
with additive indices for community- and individual-level analyses.

\begin{table}[H]
\begin{center}

\begin{tabular}{l|r|r|r|r}
\hline
  & ag\_coef & ag\_p & ind\_coef & ind\_p\\
\hline
Affect & 0.093 & 0.037 & 0.062 & 0.056\\
\hline
Insecurity & 0.015 & 0.174 & 0.030 & 0.011\\
\hline
Contact & 0.054 & 0.193 & 0.070 & 0.143\\
\hline
\end{tabular}


\caption{\label{tab:add_ind_tab}\textbf{Effect of ECPN on main outcomes with additive indices.} The first and second columns are coefficients and $p$-values for aggregate community-level analyses.  The third and fourth columns are coefficients and $p$-values for individual-level analyses.}
\end{center}
\end{table}

\hypertarget{appendix-c-placebo-analysis}{%
\subsection{Appendix C: Placebo
Analysis}\label{appendix-c-placebo-analysis}}

\hypertarget{old-appendix-c-mechanisms-and-placebo-analysis}{%
\subsection{OLD Appendix C: Mechanisms and Placebo
Analysis}\label{old-appendix-c-mechanisms-and-placebo-analysis}}

These tables show results for mechanism and placebo outcomes using
inverse-covariance weighted indices. The tables include the coefficients
and \(p\)-values for community- and individual-level analyses.

\hypertarget{appendix-x-individual-level-analysis}{%
\subsection{Appendix X: Individual-level
analysis}\label{appendix-x-individual-level-analysis}}

\begin{Shaded}
\begin{Highlighting}[]
\FunctionTok{load}\NormalTok{(}\StringTok{"../b\_creating\_outcomes/f2{-}panelData.Rdata"}\NormalTok{)}
\FunctionTok{table}\NormalTok{(panel.df}\SpecialCharTok{$}\NormalTok{community, panel.df}\SpecialCharTok{$}\NormalTok{committee, panel.df}\SpecialCharTok{$}\NormalTok{treatment)}
\end{Highlighting}
\end{Shaded}

We approximately 300 baseline respondents, roughly evenly spread
throughout all communities (between 8 and 15 respondents per community).
Participants are clustered in certain communities where they could more
easily be identified during our endline survey.

\hypertarget{appendix-d-survey-questions}{%
\subsection{Appendix D: Survey
Questions}\label{appendix-d-survey-questions}}

\textbf{Outgroup Affect}

\begin{itemize}
\tightlist
\item
  With regards to someone from {[}X GROUP{]}, would you feel
  comfortable:

  \begin{itemize}
  \tightlist
  \item
    if they worked in your field?
  \item
    paying them to watch your animals?
  \item
    trading goods with them?
  \item
    sharing a meal with them?
  \item
    with a close relative marrying a person from {[}X GROUP{]}?
  \end{itemize}
\item
  From 1-5, how much do you trust people from {[}X GROUP{]} in your
  area?
\item
  Now I'm going to ask you questions about your community here in
  Benue/Nassarawa, including {[}X GROUP{]}. Please tell me how strongly
  you agree/disagree with each of the following statements: People in
  this area can be trusted.
\end{itemize}

\textbf{Contact}

\begin{itemize}
\tightlist
\item
  Now I'm going to ask you questions about your contact with {[}X
  GROUP{]} in your area.

  \begin{itemize}
  \tightlist
  \item
    Think of the market you go to most frequently. During the past
    month, have members of X GROUP gone to that market too? In the past
    month, how many times did you interact with X group in the market?
  \end{itemize}
\item
  In the past month, have you:

  \begin{itemize}
  \tightlist
  \item
    Joined a member of X group for a social event outside the home? How
    often?
  \item
    Hosted a member of X group for a ceremony in your home? How often?
  \item
    Gone to the home of a member of X group for a ceremony? How often?
  \item
    Have you interacted with members of X group in any other way in the
    past month?
  \end{itemize}
\end{itemize}

\textbf{Insecurity}

\begin{itemize}
\tightlist
\item
  In the last year were there any areas that you avoided going to or
  through because of insecurity during the night?
\item
  In the last year were there any areas that you avoided going to or
  through because of insecurity, during the day?
\item
  In the last year, did insecurity ever prevent you from:

  \begin{itemize}
  \tightlist
  \item
    Working when you wanted to work? About how many days were you unable
    to work?
  \item
    Going to the market?
  \item
    Getting water for the household?
  \item
    Going to your field/farm?
  \item
    Moving your animals to grazing areas?
  \item
    Moving your animals to water?
  \item
    Earning money or going to work?
  \item
    Going to school?
  \end{itemize}
\end{itemize}

\textbf{Endorsement Experiment}

\begin{itemize}
\tightlist
\item
  Imagine that there is a proposal by {[}\textbf{the Farmer's
  Cooperative Society}/\textbf{MACBAN}{]} for action to enhance access
  to clean water in rural areas. Though expensive, the proposal aims to
  bring fresh, clean water to hundreds of areas without access to it,
  including this one. If this were proposed, how would you feel about
  it?
\end{itemize}

\textbf{Percent Experiemnt}

\begin{itemize}
\tightlist
\item
  Think about groups that you might join in your leisure time. Would you
  join a group that had \textbf{5/25/50/75}\% X Group members?
\item
  Think about the community you live in. Would you live in a community
  that had \textbf{5/25/50/75}\% X Group members?
\end{itemize}

\textbf{Violence Placebo}

\begin{itemize}
\tightlist
\item
  Now I am going to ask you some questions about the use of violence. Is
  it always, sometimes, rarely, or never justified to use violence to do
  each of the following:

  \begin{itemize}
  \tightlist
  \item
    Retaliate against violence
  \item
    Defend one's group
  \item
    Maintain culture and traditions
  \item
    Defend one's religion
  \item
    Bring criminals to justice
  \item
    Force the government to change their policies
  \end{itemize}
\end{itemize}

\textbf{Public Goods Game}

``Thank you very much for participating in our survey. Before I go,
there is one last thing. As you may have heard, we have development
funds to use in this community. We have randomly selected you as one of
the 50 people to receive these funds. These funds are not for a Mercy
Corps project, but rather for you to keep personally or to donate to a
community fund.

We have 1,000 Naira to give to you. It is yours, and you can use it
either way--for yourself or for a community good.

Your community and {[}joint farmer/pastoralist community{]} have created
a project committee to whom you can donate this money so that it may be
used to help both communities. The project committee has 4 people from
each community. We have found a donor that will match the funds that you
all contribute to the project committee, so that if you donate 100 Naira
the project committee receives 300 Naira, and if you donate all 1,000
Naira the project committee receives 3,000 Naira. You are welcome to
donate none, some, or all of the money to the project committee.

These are your individual donation envelopes. All the donations will be
private -- only you will know how much money you donated. It essential
that you keep how much you give private -- please do not tell anyone. I
have with me a donation envelope to collect donations. Please go into
your home, put however much of the 1,000 Naira you wish to donate to the
project committee in the envelope, take whatever amount you want to keep
for yourself, and come back to place your envelope in the donation
envelope. Remember, you are welcome to donate none, some, or all of the
money to the project committee. After that we are finished and you may
continue your day. We will come back and publicly announce how much
money your community's project committee will receive.''

\end{document}
